\Huge
Glossar
$\newline$
$\newline$
\Large
Begriffserlk\"arung $\newline$ 
\small
$\newline$ $\newline$
Waldmeister-Outdoors: $\newline$
Name der Webapp welche in dieser Projektarbeit erstellt wurde.
$\newline$ $\newline$
Waldmeister Map: $\newline$
Name der Leaflet Map, welche innerhalb der Applikation dargestellt wird.
$\newline$ $\newline$
Waldmeister App: $\newline$
Native App iOS und Android App, welche im Appstore und Android Playstore erh�ltlich ist.
$\newline$ $\newline$
Waldstandort: $\newline$
Eine Fl\"ache auf der Erdoberfl\"ache, welche einen Wald bezeichnet.
$\newline$ $\newline$
Waldstandortstyp: $\newline$
Einer von vielen Typen, welcher einen Waldstandort einen bestimmten Typ zuordnet. Meist wird hier der K\"urzel aus den Definitionen von EK72 verwendet, z.B. K\"urzel "7e". Der Name des Waldstandortstyps "7e" ist "Waldmeister-Buchenwald mit Hornstrauch" $\newline$
$\newline$ $\newline$
$\newline$ $\newline$
Vegetationslayer: $\newline$
Erster Layer oberhalb der Hintergrundkarte, welcher auf der Map dargestellt wird. Enth\"alt eingef\"arbte Polygone mit Labels welche einen Waldstandortstyp beschreiben. Die Daten, welche auf dem Vegetationslayer dargestellt werden, basieren auf der "Vegetationskundliche Kartierung der W\"alder im Kanton Z\"urich", bzw. der "Waldvegetationskarte".
Library: $\newline$
Eine Library (Programmbibliothek) bezeichnet in der Programmierung eine Sammlung von Unterprogrammen, die L�sungswege anbieten. Bibliotheken laufen im Unterschied zu Programmen nicht eigenst�ndig, sondern sie enthalten Hilfsmodule, die angefordert (importiert) werden k�nnen
$\newline$$\newline$
Framework: $\newline$
Ein Framework ist noch kein fertiges Programm. Es ist ein Rahmen (ein Ger�st) welches
der Programmierer verwenden kann um sein eigenes, pers�nliches Programm zu
erstellen

\Large
$\newline$ $\newline$
Akronyme$\newline$
\small
$\newline$
SRID = Spatial Reference Identifier
$\newline$ $\newline$
GIS = Geografisches Informationssystem
$\newline$ $\newline$
MVC = Model - View - Controller
$\newline$ $\newline$
MVVM = Model - View - ViewModel
$\newline$ $\newline$
HTML = Hypertext Markup Language
$\newline$ $\newline$
CSS = Cascading Style Sheets
$\newline$ $\newline$
ES5 = ECMA Script 5
$\newline$ $\newline$
ECMA = European Computer Manufacturers Association
$\newline$ $\newline$
SoC = Separation of Concern
$\newline$ $\newline$
SPA = Single-Page-Applikation
$\newline$ $\newline$
EK72 = Ellenberg Kl\"otzli
$\newline$ $\newline$
POI = Points of Interests
$\newline$ $\newline$
UML = Unified Modeling Language
$\newline$ $\newline$
JWT = (JSON Web Token)
$\newline$ $\newline$
JSON = JavaScript Object Notation
$\newline$ $\newline$
REST = Representational state transfer
$\newline$ $\newline$
CRUD = Create, Read, Update, Delete 
$\newline$ $\newline$
DRF = Django Rest-Framework
$\newline$ $\newline$
CI = Continuous Integration
$\newline$ $\newline$
Gdal = GDAL - Geospatial Data Abstraction Library
$\newline$ $\newline$
IFS = Institut f�r Software
$\newline$ $\newline$